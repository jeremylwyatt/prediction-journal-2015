\section{The Importance of Prediction for Manipulation}
\label{sec:motivation}
There is a large body of evidence that the human motor system uses predictive (or forward) models. These are models of the effects that motor actions have on sensory state \citep{flanagan03,flanagan06,mehta02,witney00,johansson92}. The predictions are used for a variety of purposes, including feedforward control, coordination of motor systems, action planning, and monitoring of action plan execution. Although these forward models are used in many motor systems, neuroscientists have highlighted their particular importance for human manipulation abilities:

\begin{quotation} the remarkable manipulative skill of the human hand is not the result of rapid sensorimotor processes, nor of fast or powerful effector mechanisms. Rather the secret lies in the way manual tasks are  organised and controlled by the nervous system. At the heart of this organization is prediction. Successful manipulation requires the ability both to predict the motor commands required to grasp, lift and move objects, and to predict the sensory events that arise as a consequence of these commands. --- \citep{flanagan06}
\end{quotation}

There is also evidence that predicting contact events is an essential part of the prediction process for human manipulation skills \citep{flanagan06}. This is because contact events during manipulation are used to assess intrinsic parameters of the manipulated object (friction, weight) as well as to monitor progress during the manipulative task. Each contact event also indicates the gain (loss) of a constraint on object motion. The motion behaviour of the object changes non-smoothly with such contacts. This means that to manipulate skillfully humans are likely to recruit and de-recruit  forward models of object behaviour as contacts change.  Finally there is evidence that prediction is learned before control \citep{flanagan03}, suggesting that it is necessary to learn to predict object behaviour before learning to control it.

Predictive models also have utility in robot manipulation. The dominant approach is build a model informed by mechanics \citep{mason_manipulator_1982,lynch_mechanics_1992,peshkin_motion_1988,cappelleri_designing_2006,mason_mechanics_2001,flickinger2015}, such as a rigid body mechanics engine, to make predictions of robot and object motion under contact. But these models require explicit representation, and thus precise estimation, of intrinsic parameters of the object and its surroundings such as friction, mass, mass distribution and coefficients of restitution. These are not trivial to estimate. Even if correctly estimated, rigid body simulations make approximations that can render predictions inaccurate, although methods are improving. A recent spate of work on push planning uses such models of changing contact \citep{Dogar_2010,zitoetal-iros12,Cosgun2011}, often making the quasi-static assumption to simplify planning. There is also work on using purely geometric models \citep{stillman08ijrr} for this purpose.

\def\stackalignment{l}
\begin{figure*}[t]
\centerline{\stackinset{l}{0.47in}{t}{0.16in}{(a)}{\stackinset{l}{2.0in}{t}{0.16in}{(b)}{\stackinset{l}{4.3in}{t}{0.16in}{(c)}}}\includegraphics[width=0.85\textwidth]{three-prediction-problems}}
\caption{Three types of prediction problem. A robot finger is shown in blue, objects in black, and motions of the finger as dashed lines with arrows. Top row: training actions. Bottom row: an example test action. Each column represents a different problem. Sub-figure (a): Problem 1 - Action Interpolation. Subfigure (b): Problem 2 - Transfer to novel actions. Sub-figure (c): Problem 3 - Transfer to novel shapes. \label{fig:three-prediction-problems}}
\end{figure*}

Since there is evidence that in humans the forward models are learned, it is an interesting approach to use machine learning to acquire forward models of object behaviour. The principle has been understood for a long time \citep{JordanJacobs90,JordanRumelhart92}, but the application to robotics has been limited. In this paper we propose a biologically inspired model to learn forward models of object behaviour.\footnote{We note that it is the broad architecture that is inspired by models from computational neuroscience, whereas the specific learning algorithms are drawn from statistical machine learning, and thus make no claims to biological inspiration.}Although there are many papers on learning forward models, none of the extant work in robotics concerns the general case of making precise (metric) predictions of object motion, when the object is manipulated, and can have changing contacts with the environment.