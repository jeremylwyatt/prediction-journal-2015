
\section{Conclusions}\label{sec:Discussion}

This paper has made several findings: that modular predictors of object motion can be learned; that learning transfer is possible; that contact information assists transfer; that factorisation helps exploit this information; and that statistical machine learning can match---or for specific objects exceed---tuned physics engine performance. The paper presented the first results on real objects for object transfer, and the first presentation of our modular-products scheme for prediction. What do these results tell us about the way to proceed? What is the space of methods for prediction? What promising avenues are there? We note the following issues.

\subsubsection{ Prior knowledge} While the prior knowledge embodied by classical mechanics provides generality, the necessary approximations made in implementations can hinder accurate prediction. Rigid body simulators also require learning of the intrinsic parameters of the object, but sometimes have too many constraints to wrap themselves finely around real data. One the other hand it is clear that some structural knowledge is required: contact information is structural knowledge benefiting transfer. Pure tabula rasa learning is unlikely to be the answer.

\subsubsection{ Local shape} The learners employed here used much less information than the full object shape. Further shape information might improve prediction further. Specifically, the local surface shapes of both surfaces at a contact influence object motion. Experts specialised to local shape contexts may improve prediction performance. In the scheme presented here, this would result in nesting another modular structure inside the product of experts. It would also provide a means to solve the problem of how to automatically attach experts to objects.

\subsubsection{ Modularity} There is evidence from neuroscience tha the brain employs modularity in prediction, and in developing expert motor skills. In this paper we have argued that this is a promising way to proceed for robotics. Rather than learning a general purpose predictor, why not learn very many, very specific predictors? Memory is in current computing technology is cheap, and so learning many hundreds or even thousands of object specific prediction modules is feasible. Modularity is part of the way to proceed.

\subsubsection{Multiple changing contacts}  In dextrous manipulation, the hand makes multiple, changing contacts with the object. Prediction for manipulation must therefore account for these non-smooth changes. Hybrid models may be a way to proceed. These have been explored in modelling the dynamics of changing contact in walking, but have yet to be applied to manipulation.

\subsubsection{ Training noise} Transfer performance in the approach presented here degrades under training noise. Recently, we have partially addressed this by removing noise at prediction time using kinematic optimisation \cite{belter2014iros}. This combines the benefits of collision detection with the benefits of machine learning, and improves prediction performance significantly. It is, however, unclear as to whether the learned models are then transferrable to novel objects or actions. Thus, whether this approach is extensible is an open question. 

\subsubsection{Dynamics} We have restricted this study to quasi-static cases, but the formulation of the basic regression problem with dynamics was given. Learning with dynamics is the next obvious step. 


%This paper has sought to establish a case for modular, machine learning approaches to metric prediction as a promising alternative to analytic modelling. Essentially machine learning plus modularisation allows us to predict accurately in the face of unobservable parameters. Unlike other learning approaches, ours provides precise predictions of rigid body motion over many steps, and can transfer predictions to novel actions and objects. This required exploiting the insight from analytic modelling that each contact must be modelled explicitly, and the structure of the learner should reflect the contact structure so as to to exploit this information. In summary combining insights from analytic and machine learning approaches is, we believe, the way forward.
