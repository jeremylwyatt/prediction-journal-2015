%%%%%%%%%%%%%%%%%%%%%%% file template.tex %%%%%%%%%%%%%%%%%%%%%%%%%
%
% This is a general template file for the LaTeX package SVJour3
% for Springer journals.          Springer Heidelberg 2010/09/16
%
% Copy it to a new file with a new name and use it as the basis
% for your article. Delete % signs as needed.
%
% This template includes a few options for different layouts and
% content for various journals. Please consult a previous issue of
% your journal as needed.
%
%%%%%%%%%%%%%%%%%%%%%%%%%%%%%%%%%%%%%%%%%%%%%%%%%%%%%%%%%%%%%%%%%%%
%
% First comes an example EPS file -- just ignore it and
% proceed on the \documentclass line
% your LaTeX will extract the file if required
%\begin{filecontents*}{example.eps}
%!PS-Adobe-3.0 EPSF-3.0
%%BoundingBox: 19 19 221 221
%%CreationDate: Mon Sep 29 1997
%%Creator: programmed by hand (JK)
%%EndComments
%gsave
%newpath
%  20 20 moveto
%  20 220 lineto
%  220 220 lineto
%  220 20 lineto
%closepath
%2 setlinewidth
%gsave
%  .4 setgray fill
%grestore
%stroke
%grestore
%\end{filecontents*}
%
\RequirePackage{fix-cm}
%
%\documentclass{svjour3}                     % onecolumn (standard format)
%\documentclass[smallcondensed]{svjour3}     % onecolumn (ditto)
%\documentclass[smallextended]{svjour3}       % onecolumn (second format)
\documentclass[natbib,twocolumn]{svjour3}          % twocolumn
%
\smartqed  % flush right qed marks, e.g. at end of proof
%
\usepackage{graphicx}
%
% \usepackage{mathptmx}      % use Times fonts if available on your TeX system
%
% insert here the call for the packages your document requires
%\usepackage{latexsym}
% etc.
\usepackage{amsmath,amssymb,amsbsy}
\usepackage{algorithm}
\usepackage{graphicx}
\usepackage{epstopdf}
\usepackage{algorithmic}
\usepackage{stackengine}

%
% please place your own definitions here and don't use \def but
% \newcommand{}{}
\newcommand{\Ex}{\mathop{\mathbb E\/}}
\newcommand{\argmax}[1]{\underset{#1}{\operatorname{argmax}}\medspace}

% Insert the name of "your journal" with
% \journalname{myjournal}
%
\begin{document}

\title{Learning modular and transferable forward models of the motions of push manipulated objects\thanks{Kopicki is identified as the first author, and Wyatt, Kopicki and Zurek are identified as the primary authors. Wyatt is corresponding author. We gratefully acknowledge support of grant EU-FP7-IST-600918-PaCMan.}
}
%\subtitle{Do you have a subtitle?\\ If so, write it here}

\titlerunning{Learning Modular and Transferable Forward Models of the Motions of Push Manipulated Objects }        % if too long for running head

\author{Marek Kopicki  \and Sebastian Zurek \and Rustam Stolkin \and Thomas Moerwald \and Jeremy L. Wyatt
}

\authorrunning{Kopicki et al.} % if too long for running head

\institute{Kopicki, Wyatt et al.\at
              School of Computer Science \\
              University of Birmingham \\
              Edgbaston \\
              B15 2TT
              Tel.: +44-121-414-4788\\
              \email{\{msk,jlw\}@cs.bham.ac.uk}           \\
              \and
              Thomas Moerwald \at
              ACIN, TU Wien \\
              Gusshausstrasse 27-29 / E376
1040 Vienna, Austria \\
Tel.: +43 (1)58801 - 376631 \\
              \email{moerwald@acin.tuwien.ac.at}           %  \\
%             \emph{Present address:} of F. Author  %  if needed
}
\date{Received: date / Accepted: date}
% The correct dates will be entered by the editor
\maketitle

\begin{abstract}
Human manipulation relies on an ability to predict object behaviour during manipulation. Humans learn predictors which can be tuned for specific objects. This paper presents a prediction scheme inspired by models from computational neuroscience. First, we formulate the prediction problem for the quasi-static case, and in two different machine learning frameworks: i) regression and ii) density estimation. The prediction architecture is modular: many simple object and context specific predictors are learned. We describe the kinds of contact information necessary to predict, and show that object specific predictors typically outperform a tuned rigid body simulator. We then extend the density estimation approach to transfer learning, using a factored representation of contacts. This formulation permits transfer of learning to objects of novel shape, and to novel actions. We show empirically that transfer learning can match the performance of physics simulation, but only when using a factored model, and information on all contact relations in which the object participates.
\keywords{Transfer learning \and Manipulation \and Prediction \and Robot Learning}
% \PACS{PACS code1 \and PACS code2 \and more}
% \subclass{MSC code1 \and MSC code2 \and more}
\end{abstract}

\section{Introduction}\label{sec:Introduction}

Predicting what will happen next is central to intelligent behaviour \citep{craik1967nature}. If an agent cannot predict the effects of its actions it cannot autonomously plan a course of action to achieve its goals. Modelling a robot's interactions with the world, so that it can make useful predictions about the effects of its actions, is a challenging set of problems. This paper is about how to predict the motions of a rigid object when manipulated by a robot. While simple to pose, this problem is not easy to solve. This paper argues that predicting future behaviour on the basis of past experience is unavoidably a problem of learning. It also shows such predictors can be learned, and how the learned models can be transferred to new objects and actions. The paper focuses on kinematic models of object behaviour, since forces and masses are assumed to be unknown at prediction time. Within this scope the paper makes the following contributions. First it presents a {\em modular machine learning} solution to object motion prediction problems, and shows for the first time that when tuned for specific objects (or contexts) this outperforms physics simulation on a variety of real objects.  Second, it shows for the first time results for transfer learning on real objects. Transferring predictions to novel objects is a much harder problem than the first problem of context specific prediction. This generality is precisely what physics engines, such as rigid body simulators, are designed to achieve. In this paper, we show that learning---if and only if good representations are chosen---can be used to achieve transfer, and even approach the prediction performance of a rigid body simulation on novel objects and actions.

This paper thus extends our previous work \citep{kopicki_prediction_2010,kopicki-etal-icra11}  where the core prediction algorithm was presented, and tested mostly in simulation.  This paper tests three specific hypotheses, all evaluated with respect to real objects. Hypothesis~1 is that a {\em modular learning} approach can outperform physics engines for prediction of rigid body motion.  Hypothesis~2 is that by factorising these modular predictors they can be transferred to make predictions about novel actions. Hypothesis~3 is that by factorising, learning can be transferred  to make predictions about novel shapes. For hypotheses 2 and 3 we suppose that only for some representations is learning transfer effective.

The paper is structured as follows.  Section~\ref{sec:motivation} describes why the general problem of prediction for manipulation is important, and also explains versions of it which are unsolved. Section~\ref{sec:schema} describes the specific problems tackled in this paper in an informal style, and argues for the benefits of a modular learning approach to solving it. Next, Section~\ref{sec:Representations} introduces representations of object motion, enabling a formal problem statement in Section~\ref{sec:PredictionProblem} and formulation in both regression and density estimation frameworks. Section~\ref{sec:InfoForPrediction} incorporates contact information, and Section~\ref{sec:Factors} describes how we can factor the learner by the contacts to achieve transfer learning. Section~\ref{sec:Implementation} gives implementation details, Section~\ref{sec:Experiment} the experimental method, and Section~\ref{sec:Results}
the corresponding results. Section~\ref{sec:Background} reviews related work.  We finish with a discussion in Section~\ref{sec:Discussion}.
\section{The Importance of Prediction for Manipulation}
\label{sec:motivation}
There is a large body of evidence that the human motor system uses predictive (or forward) models. These are models of the effects that motor actions have on sensory state \citep{flanagan03,flanagan06,mehta02,witney00,johansson92}. The predictions are used for a variety of purposes, including feedforward control, coordination of motor systems, action planning, and monitoring of action plan execution. Although these forward models are used in many motor systems, neuroscientists have highlighted their particular importance for human manipulation abilities:

\begin{quotation} the remarkable manipulative skill of the human hand is not the result of rapid sensorimotor processes, nor of fast or powerful effector mechanisms. Rather the secret lies in the way manual tasks are  organised and controlled by the nervous system. At the heart of this organization is prediction. Successful manipulation requires the ability both to predict the motor commands required to grasp, lift and move objects, and to predict the sensory events that arise as a consequence of these commands. --- \citep{flanagan06}
\end{quotation}

There is also evidence that predicting contact events is an essential part of the prediction process for human manipulation skills \citep{flanagan06}. This is because contact events during manipulation are used to assess intrinsic parameters of the manipulated object (friction, weight) as well as to monitor progress during the manipulative task. Each contact event also indicates the gain (loss) of a constraint on object motion. The motion behaviour of the object changes non-smoothly with such contacts. This means that to manipulate skillfully humans are likely to recruit and de-recruit  forward models of object behaviour as contacts change.  Finally there is evidence that prediction is learned before control \citep{flanagan03}, suggesting that it is necessary to learn to predict object behaviour before learning to control it.

Predictive models also have utility in robot manipulation. The dominant approach is build a model informed by mechanics \citep{mason_manipulator_1982,lynch_mechanics_1992,peshkin_motion_1988,cappelleri_designing_2006,mason_mechanics_2001,flickinger2015}, such as a rigid body mechanics engine, to make predictions of robot and object motion under contact. But these models require explicit representation, and thus precise estimation, of intrinsic parameters of the object and its surroundings such as friction, mass, mass distribution and coefficients of restitution. These are not trivial to estimate. Even if correctly estimated rigid body simulations make approximations that can render predictions inaccurate, although methods are improving. A recent spate of work on push planning uses such models of changing contact \citep{Dogar_2010,zitoetal-iros12,Cosgun2011}, often making the quasi-static assumption to simplify planning. There is also work on using purely geometric models \citep{stillman08ijrr} for this purpose.

Since there is evidence that in humans the forward models are learned, it is an interesting approach to use machine learning to acquire forward models of object behaviour. The principle has been understood for a long time \citep{JordanJacobs90,JordanRumelhart92}, but the application to robotics has been limited. In this paper we propose a biologically inspired model to learn forward models of object behaviour.\footnote{We note that it is the broad architecture that is inspired by models from computational neuroscience, whereas the specific learning algorithms are drawn from statistical machine learning, and thus make no claims to biological inspiration.}Although there are many papers on learning approaches to learning forward models, none of the extant work in robotics concerns the general case of making precise (metric) predictions of object motion, when the object is manipulated, and can have changing contacts with the environment.
\section{Three Prediction Problems	}
\label{sec:schema}

\def\stackalignment{l}
\begin{figure*}[t]
\centerline{\stackinset{l}{0.47in}{t}{0.16in}{(a)}{\stackinset{l}{2.0in}{t}{0.16in}{(b)}{\stackinset{l}{4.3in}{t}{0.16in}{(c)}}}\includegraphics[width=0.85\textwidth]{three-prediction-problems}}
\caption{Three types of prediction problem. A robot finger is shown in blue, objects in black, and motions of the finger as dashed lines with arrows. Top row: training actions. Bottom row: an example test action. Each column represents a different problem. Sub-figure (a): Problem 1 - Action Interpolation. Subfigure (b): Problem 2 - Transfer to novel actions. Sub-figure (c): Problem 3 - Transfer to novel shapes. \label{fig:three-prediction-problems}}

\end{figure*}

To understand hypotheses H1-H3 consider three corresponding prediction problems in Figure~\ref{fig:three-prediction-problems}(a)-(c): action interpolation (P1), action transfer (P2) and shape transfer (P3). Consider how each might be tackled using either: (i) rigid body simulator employing classical mechanics or (ii) statistical machine learning. Assume object and environment shape to be known.

\subsection{Action Interpolation (P1).} Suppose you have seen some pushes of an object (Figure~\ref{fig:three-prediction-problems} (a) top row). In some cases the object tipped, in some it slid. Now you are presented with a new push direction (bottom row, left column). The task is to predict the new object motion. To perform this using rigid body mechanics requires knowledge of object and environment parameters such as the mass and frictional coefficients. These could be inferred by fitting the previous motions via parameter estimation for the rigid body simulator. Thus, even for a classical mechanics approach, the problem involves learning. Alternatively generalisation across actions is feasible using semi- or non-parametric machine learning. This is because the experiences span the test case: there aren't any exactly similar actions, but there are many with similar features. Hence this problem involves {\em action interpolation}.

\subsection{Action Transfer (P2).} Figure~\ref{fig:three-prediction-problems} (b) depicts a harder problem since the test action (bottom row) now sits outside the range of training actions. Hence this problem is known as {\em action transfer}. Turning the object around, since it is not symmetric, means that the effects of actions are quite different to before. For example, pushing the top of the L-shaped object will no longer induce it to tip over. This is because the horizontal flap cannot pass through the table: it provides a {\em kinematic constraint} on the motion of the object. This makes action transfer problems challenging for tabula rasa machine learning. Such problems should, however, be no more challenging for rigid body simulation than problem P1, since once an object's parameters are estimated, the rigid body simulator can produce predictions for any action.

\subsection{Shape Transfer (P3).} Finally (Figure~\ref{fig:three-prediction-problems} (c) top row) requires generalising predictions about action effects to novel shapes. The training data consists of pushes of two objects of different shape. The test action is a push of a previously unseen object of quite different global shape to the training objects. This is challenging for tabula rasa machine learning because small changes in object shape can lead to large changes in behaviour. The problem is also challenging for an approach using a tuned rigid body simulator , since since estimation for mass and frictional coefficients for the test object, must be based on the estimates made from the training data, and will thus be sensitive to estimation errors.

\subsection{The case for modular prediction learning}

Rigid body simulation is an initially appealing solution to these problems. Rigid body simulators can produce precise, physically plausible predictions about the motion of an object, including predictions for objects of novel shape and for novel actions. Unfortunately it quickly becomes apparent that this creates as well as solves problems. First, rigid body simulators require knowledge of many parameters intrinsic to the object and environment: frictional coefficients, coefficients of restitution, mass, mass distribution etc. These must all be estimated quite precisely from training data to produce an accurate prediction. Thus rigid body simulation does not eliminate the need for learning. Worse, it requires estimation of parameters that are not easily measured by a robot. Finally, rigid body simulators necessarily use approximate models of phenomena such as friction, and this leads to inaccurate predictions. While this paper is concerned with rigid bodies only, these problems only become worse when the scope is widened to include deformable objects and liquids.
\begin{figure}[t]
\centerline{\includegraphics[width=0.9\columnwidth]{modular-schema}}
\caption{A modular prediction scheme for solving problem P1. Visual object identification selects a context/predictor, and gives it the object pose and intended finger trajectory as initial input. Prediction is fed back on itself to produce a multi-step prediction. \label{fig:modular-simple}}
\end{figure}
A different approach is required. Tabula rasa learning is one alternative, but there are problems.  To begin with suppose we try to learn a single predictor from data. The aim would be to predict action effects across a wide range of object shapes and materials. This is hard because the a great deal of variation and complexity must be captured in a single learner. A clue as to a different way to proceed instead comes from computational neuroscience, where models such as MOSAIC employ modular prediction \citep{Haruno_MOSAIC_2008}. Modular means that the overall prediction engine consists of many context specific predictors (Figure~\ref{fig:modular}), where a context is an object, or an object-environment combination. The first advantage of this is that it can be easier to solve many simple learning problems than one complex learning problem. Second, unobservable parameters (frictional coefficients, mass, mass distribution) need not be modelled explicitly but are instead captured implicitly by being associated with a particular context. Whereas MOSAIC couples control and prediction, it avoids real objects (working with simulated mass spring systems). Our work focuses on pure prediction, but for real objects. Our modular prediction scheme uses vision to distinguish the context, by identifying the object shape.

Having explained our overall scheme, we now turn to the mathematical details of how to model robot-object-environment interactions, which will lead in turn to posing the three prediction problems formally.

\input{learning}
\input{implementation}
\input{results}
\section{Related Work}\label{sec:Background}

Related work is split into four broad areas: neuroscience, analytic approaches, qualitative physics and machine learning. Prediction of motor effects on the body has long been studied in neuroscience \citep{Miall1996,flanagan03}.  MOSAIC was an early computational model of prediction and control in the cerebellum using a modular scheme \citep{Haruno_MOSAIC_2008}, where predictions can be made by convex combinations of learned predictors. Other bio-inspired modular prediction schemes were independently derived by roboticists \citep{demiris2006hierarchical}. These models all differ from ours in that our work is the first attempt at modelling the motions of objects with kinematic constraints. There is also evidence that infants can learn object specific motions \citep{Bahrick1995}. It is also clear that while some object knowledge may be innate \citep{spelke1994early}, object specific predictions must be learned, and are critical to our manipulation skills \citep{flanagan06}. So in general terms modular learning of predictions of object behaviour is cognitively plausible.

There is substantial work in robotics on classical analytic mechanics models of pushing \citep{mason_manipulator_1982,lynch_mechanics_1992,peshkin_motion_1988,cappelleri_designing_2006}, on both kinematic and dynamic models of manipulation effects \citep{mason_mechanics_2001}. Such analytic models are good predictors if their key parameters (e.g. friction) are precisely known. They can also inform push planning under pose uncertainty \citep{brost1985planning}. There is a separate body of work on qualitative models of action effects on objects, rooted in naive physics \citep{hayes1995second}, and qualitative physics \citep{kuipers1986qualitative}. In a similar spirit there is work on using physics engines to learn qualitative action effects \citep{Mugan-tamd-12}, and high level planning of manipulation \citep{stillman08ijrr,roy2004mental} using qualitative action models. Some early ideas on push planning have reappeared in recent robots that plan pushes to enable grasps in clutter \citep{Dogar_2010}.

At the other end learning approaches have been used. Some have been used to model behaviour without changes in contact, e.g. predicting the motion of an object, robot arm or gripper in free space \citep{Ting06,Boots14,dearden2005learning}. Other work concerns learning the dynamics of an object with a single, constant contact (such as pole balancing) \citep{Schaal97,SchaalAtkeson97}. Finally there has been work on affordance learning in which the system learns which variables are relevant to predicting object motion \citep{montesano08,moldovan12,hermans11,fitzpatrick_learning_2003,ridge2010self,kroemer2014,hermans13}. The restriction of each of these papers is that they make qualitative predictions of object motion, such as a classification of the type of motion outcome. Finally there has been recent work in which metric motions is learned from experience. Stoytchev \citep{Stoytchev_affordances_2008} enables a robot to learn action effects of sticks and hook-like tools by pushing objects. This work simplifies the domain by using circular pucks as objects, and four planar motions as actions. Action outcomes were learned for various tools in a modular fashion, but without transfer learning.  In \citep{mericli2014} the metric planar motion of pushed objects on the plane is learned, but the learning is restricted to motion in free space. 

Our work thus sits at the intersection of some these approaches. We embrace machine learning and modularity to achieve scalability, but we also explicitly model each contact constraint. Our machine leanring approach is used to make metrically precise predictions, but also under contact, including changing contact with the environment. In this way we try to re-achieve in a machine learning framework what only the analytic approach has attempted to date: metric prediction of motion transferable to novel actions and objects. We avoid analytic modelling, instead combining modelling of kinematic constraints with machine learning based approaches to provide a hybrid solution to prediction for manipulation.

\section{Conclusions}\label{sec:Discussion}

This paper has made several findings: that modular predictors of object motion can be learned; that learning transfer is possible; that contact information assists transfer; that factorisation helps exploit this information; and that statistical machine learning can match---or for specific objects exceed---tuned physics engine performance. The paper presented the first results on real objects for object transfer, and the first presentation of our modular-products scheme for prediction. What do these results tell us about the way to proceed? What is the space of methods for prediction? What promising avenues are there? We note the following issues.

\subsubsection{ Prior knowledge} While the prior knowledge embodied by classical mechanics provides generality, the necessary approximations made in implementations can hinder accurate prediction. Rigid body simulators also require learning of the intrinsic parameters of the object, but sometimes have too many constraints to wrap themselves finely around real data. One the other hand it is clear that some structural knowledge is required: contact information is structural knowledge benefiting transfer. Pure tabula rasa learning is unlikely to be the answer.

\subsubsection{ Local shape} The learners employed here used much less information than the full object shape. Further shape information might improve prediction further. Specifically, the local surface shapes of both surfaces at a contact influence object motion. Experts specialised to local shape contexts may improve prediction performance. In the scheme presented here, this would result in nesting another modular structure inside the product of experts. It would also provide a means to solve the problem of how to automatically attach experts to objects.

\subsubsection{ Modularity} There is evidence from neuroscience tha the brain employs modularity in prediction, and in developing expert motor skills. In this paper we have argued that this is a promising way to proceed for robotics. Rather than learning a general purpose predictor, why not learn very many, very specific predictors? Memory is in current computing technology is cheap, and so learning many hundreds or even thousands of object specific prediction modules is feasible. Modularity is part of the way to proceed.

\subsubsection{Multiple changing contacts}  In dextrous manipulation, the hand makes multiple, changing contacts with the object. Prediction for manipulation must therefore account for these non-smooth changes. Hybrid models may be a way to proceed. These have been explored in modelling the dynamics of changing contact in walking, but have yet to be applied to manipulation.

\subsubsection{ Training noise} Transfer performance in the approach presented here degrades under training noise. Recently, we have partially addressed this by removing noise at prediction time using kinematic optimisation \cite{belter2014iros}. This combines the benefits of collision detection with the benefits of machine learning, and improves prediction performance significantly. It is, however, unclear as to whether the learned models are then transferrable to novel objects or actions. Thus, whether this approach is extensible is an open question. 

\subsubsection{Dynamics} We have restricted this study to quasi-static cases, but the formulation of the basic regression problem with dynamics was given. Learning with dynamics is the next obvious step. 


%This paper has sought to establish a case for modular, machine learning approaches to metric prediction as a promising alternative to analytic modelling. Essentially machine learning plus modularisation allows us to predict accurately in the face of unobservable parameters. Unlike other learning approaches, ours provides precise predictions of rigid body motion over many steps, and can transfer predictions to novel actions and objects. This required exploiting the insight from analytic modelling that each contact must be modelled explicitly, and the structure of the learner should reflect the contact structure so as to to exploit this information. In summary combining insights from analytic and machine learning approaches is, we believe, the way forward.

%\input{appendix}
%\input{acknowledgements}

%\section{Section title}
%\label{sec:1}
%Text with citations \cite{RefB} and \cite{RefJ}.
%\subsection{Subsection title}
%\label{sec:2}
%as required. Don't forget to give each section
%and subsection a unique label (see Sect.~\ref{sec:1}).
%\paragraph{Paragraph headings} Use paragraph headings as needed.
%\begin{equation}
%a^2+b^2=c^2
%\end{equation}

% For one-column wide figures use
%\begin{figure}
%% Use the relevant command to insert your figure file.
%% For example, with the graphicx package use
%  \includegraphics{example.eps}
%% figure caption is below the figure
%\caption{Please write your figure caption here}
%\label{fig:1}       % Give a unique label
%\end{figure}
%
% For two-column wide figures use
%\begin{figure*}
%% Use the relevant command to insert your figure file.
%% For example, with the graphicx package use
%  \includegraphics[width=0.75\textwidth]{example.eps}
%% figure caption is below the figure
%\caption{Please write your figure caption here}
%\label{fig:2}       % Give a unique label
%\end{figure*}
%
% For tables use
%\begin{table}
%% table caption is above the table
%\caption{Please write your table caption here}
%\label{tab:1}       % Give a unique label
%% For LaTeX tables use
%\begin{tabular}{lll}
%\hline\noalign{\smallskip}
%first & second & third  \\
%\noalign{\smallskip}\hline\noalign{\smallskip}
%number & number & number \\
%number & number & number \\
%\noalign{\smallskip}\hline
%\end{tabular}
%\end{table}

%\begin{acknowledgements}
%If you'd like to thank anyone, place your comments here
%and remove the percent signs.
%\end{acknowledgements}

% BibTeX users please use one of
%\bibliographystyle{spbasic}      % basic style, author-year citations
%\bibliographystyle{spmpsci}      % mathematics and physical sciences
%\bibliographystyle{spphys}       % APS-like style for physics
\bibliographystyle{spmpscinat}
\bibliography{main}   % name your BibTeX data base

% Non-BibTeX users please use
%\begin{thebibliography}{}
%
% and use \bibitem to create references. Consult the Instructions
% for authors for reference list style.
%
%\bibitem{RefJ}
%% Format for Journal Reference
%Author, Article title, Journal, Volume, page numbers (year)
%% Format for books
%\bibitem{RefB}
%Author, Book title, page numbers. Publisher, place (year)
%% etc
%\end{thebibliography}

\end{document}
% end of file template.tex

