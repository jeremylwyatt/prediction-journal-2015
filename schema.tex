\section{Three Prediction Problems	}
\label{sec:schema}

\def\stackalignment{l}
\begin{figure*}[t]
\centerline{\stackinset{l}{0.47in}{t}{0.16in}{(a)}{\stackinset{l}{2.0in}{t}{0.16in}{(b)}{\stackinset{l}{4.3in}{t}{0.16in}{(c)}}}\includegraphics[width=0.85\textwidth]{three-prediction-problems}}
\caption{Three types of prediction problem. A robot finger is shown in blue, objects in black, and motions of the finger as dashed lines with arrows. Top row: training actions. Bottom row: an example test action. Each column represents a different problem. Sub-figure (a): Problem 1 - Action Interpolation. Subfigure (b): Problem 2 - Transfer to novel actions. Sub-figure (c): Problem 3 - Transfer to novel shapes. \label{fig:three-prediction-problems}}

\end{figure*}

To understand hypotheses H1-H3 consider three corresponding prediction problems in Figure~\ref{fig:three-prediction-problems}(a)-(c): action interpolation (P1), action transfer (P2) and shape transfer (P3). Consider how each might be tackled using either: (i) rigid body simulator employing classical mechanics or (ii) statistical machine learning. Assume object and environment shape to be known.

\subsection{Action Interpolation (P1).} Suppose you have seen some pushes of an object (Figure~\ref{fig:three-prediction-problems} (a) top row). In some cases the object tipped, in some it slid. Now you are presented with a new push direction (bottom row, left column). The task is to predict the new object motion. To perform this using rigid body mechanics requires knowledge of object and environment parameters such as the mass and frictional coefficients. These could be inferred by fitting the previous motions via parameter estimation for the rigid body simulator. Thus, even for a classical mechanics approach, the problem involves learning. Alternatively generalisation across actions is feasible using semi- or non-parametric machine learning. This is because the experiences span the test case: there aren't any exactly similar actions, but there are many with similar features. Hence this problem involves {\em action interpolation}.

\subsection{Action Transfer (P2).} Figure~\ref{fig:three-prediction-problems} (b) depicts a harder problem since the test action (bottom row) now sits outside the range of training actions. Hence this problem is known as {\em action transfer}. Turning the object around, since it is not symmetric, means that the effects of actions are quite different to before. For example, pushing the top of the L-shaped object will no longer induce it to tip over. This is because the horizontal flap cannot pass through the table: it provides a {\em kinematic constraint} on the motion of the object. This makes action transfer problems challenging for tabula rasa machine learning. Such problems should, however, be no more challenging for rigid body simulation than problem P1, since once an object's parameters are estimated, the rigid body simulator can produce predictions for any action.

\subsection{Shape Transfer (P3).} Finally (Figure~\ref{fig:three-prediction-problems} (c) top row) requires generalising predictions about action effects to novel shapes. The training data consists of pushes of two objects of different shape. The test action is a push of a previously unseen object of quite different global shape to the training objects. This is challenging for tabula rasa machine learning because small changes in object shape can lead to large changes in behaviour. The problem is also challenging for an approach using a tuned rigid body simulator , since since estimation for mass and frictional coefficients for the test object, must be based on the estimates made from the training data, and will thus be sensitive to estimation errors.

\subsection{The case for modular prediction learning}

Rigid body simulation is an initially appealing solution to these problems. Rigid body simulators can produce precise, physically plausible predictions about the motion of an object, including predictions for objects of novel shape and for novel actions. Unfortunately it quickly becomes apparent that this creates as well as solves problems. First, rigid body simulators require knowledge of many parameters intrinsic to the object and environment: frictional coefficients, coefficients of restitution, mass, mass distribution etc. These must all be estimated quite precisely from training data to produce an accurate prediction. Thus rigid body simulation does not eliminate the need for learning. Worse, it requires estimation of parameters that are not easily measured by a robot. Finally, rigid body simulators necessarily use approximate models of phenomena such as friction, and this leads to inaccurate predictions. While this paper is concerned with rigid bodies only, these problems only become worse when the scope is widened to include deformable objects and liquids.
\begin{figure}[t]
\centerline{\includegraphics[width=0.9\columnwidth]{modular-schema}}
\caption{A modular prediction scheme for solving problem P1. Visual object identification selects a context/predictor, and gives it the object pose and intended finger trajectory as initial input. Prediction is fed back on itself to produce a multi-step prediction. \label{fig:modular-simple}}
\end{figure}
A different approach is required. Tabula rasa learning is one alternative, but there are problems.  To begin with suppose we try to learn a single predictor from data. The aim would be to predict action effects across a wide range of object shapes and materials. This is hard because the a great deal of variation and complexity must be captured in a single learner. A clue as to a different way to proceed instead comes from computational neuroscience, where models such as MOSAIC employ modular prediction \citep{Haruno_MOSAIC_2008}. Modular means that the overall prediction engine consists of many context specific predictors (Figure~\ref{fig:modular}), where a context is an object, or an object-environment combination. The first advantage of this is that it can be easier to solve many simple learning problems than one complex learning problem. Second, unobservable parameters (frictional coefficients, mass, mass distribution) need not be modelled explicitly but are instead captured implicitly by being associated with a particular context. Whereas MOSAIC couples control and prediction, it avoids real objects (working with simulated mass spring systems). Our work focuses on pure prediction, but for real objects. Our modular prediction scheme uses vision to distinguish the context, by identifying the object shape.

Having explained our overall scheme, we now turn to the mathematical details of how to model robot-object-environment interactions, which will lead in turn to posing the three prediction problems formally.
