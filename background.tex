\section{Related Work}\label{sec:Background}

Related work is split into four broad areas: neuroscience, analytic approaches, qualitative physics and machine learning. Prediction of motor effects on the body has long been studied in neuroscience \citep{Miall1996,flanagan03}.  MOSAIC was an early computational model of prediction and control in the cerebellum using a modular scheme \citep{Haruno_MOSAIC_2008}, where predictions can be made by convex combinations of learned predictors. Other bio-inspired modular prediction schemes were independently derived by roboticists \citep{demiris2006hierarchical}. These models all differ from ours in that our work is the first attempt at learning to model the motions of objects with kinematic constraints. There is also evidence that infants can learn object specific motions \citep{Bahrick1995}. It is also clear that while some object knowledge may be innate \citep{spelke1994early}, object specific predictions must be learned, and are critical to our manipulation skills \citep{flanagan06}. So in general terms modular learning of predictions of object behaviour is cognitively plausible.

There is substantial work in robotics on classical analytic mechanics models of pushing \citep{mason_manipulator_1982,lynch_mechanics_1992,peshkin_motion_1988,cappelleri_designing_2006}, on both kinematic and dynamic models of manipulation effects \citep{mason_mechanics_2001}. Such analytic models are good metric predictors if their key parameters (e.g. friction) are precisely known, although qualitative predictions are robust to parameter uncertainty. They can also inform push planning under pose uncertainty \citep{brost1985planning}. There is a separate body of work on qualitative models of action effects on objects, rooted in naive physics \citep{hayes1995second}, and qualitative physics \citep{kuipers1986qualitative}. In a similar spirit there is work on using physics engines to learn qualitative action effects \citep{Mugan-tamd-12}, and high level planning of manipulation \citep{stillman08ijrr,roy2004mental} using qualitative action models. Some early ideas on push planning have reappeared in recent robots that plan pushes to enable grasps in clutter \citep{Dogar_2010}.

At the other end of the spectrum, learning approaches have been used. Some have been used to model behaviour without changes in contact, e.g. predicting the motion of an object, robot arm or gripper in free space \citep{Ting06,Boots14,dearden2005learning}. Other work concerns learning the dynamics of an object with a single, constant contact (such as pole balancing) \citep{Schaal97,SchaalAtkeson97}. Finally there has been work on affordance learning in which the system learns which variables are relevant to predicting object motion \citep{montesano08,moldovan12,hermans11,fitzpatrick_learning_2003,ridge2010self,kroemer2014}. The restriction of each of these papers is that they make qualitative predictions of object motion, such as a classification of the type of motion outcome. There has also been work on predicting stable push locations \citep{hermans13}. Finally there has been recent work in which metric motion models are learned from experience. Stoytchev \citep{Stoytchev_affordances_2008} enabled a robot to learn action effects of sticks and hook-like tools by pushing objects. This work simplifies the domain by using circular pucks as objects, and four planar motions as actions. Action outcomes were learned for various tools in a modular fashion, but without transfer learning.  In \citep{mericli2014} the metric planar motion of pushed objects on the plane is learned. 

Our work thus sits at the intersection of some these approaches. We embrace machine learning and modularity to achieve scalability, but we also explicitly model each contact constraint. Our machine leanring approach is used to make metrically precise predictions, but also under contact, including changing contact with the environment. In this way we try to re-achieve in a machine learning framework what only the analytic approach has attempted to date: metric prediction of motion transferable to novel actions and objects. We avoid analytic modelling, instead combining modelling of kinematic constraints with machine learning based approaches to provide a hybrid solution to prediction for manipulation.